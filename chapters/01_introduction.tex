\section{Introduction}
\label{sec:introduction}

While quantum computing is still at an early stage, many quantum algorithms seem to provide drastic accelerations compared to their current classical counterparts.
One of the best known of these is Shor's algorithm which is almost exponentially faster than the currently most efficient factoring algorithms and thus poses a threat to cryptosystems that rely on the factoring problem being hard to solve~\cite{Shor}.

However building a quantum computer and making use of it is not an easy task.
It is essential for such a quantum system to be isolated from unknown interactions with the external world which may mess up the internal state and thus the received results~\cite[p. 34]{Milburn}.
Simultaneously those systems need to be manipulable to an extent in order to be able to implement quantum gates or encoding starting states for the individual qubits~\cite{Matuschak2019}.

Qubits on their own, as well as in an entangled state are error prone in the presence of \emph{noise} and \emph{decoherence}~\cite[p. 34]{Milburn}.
Both appear even with current quantum systems.
To protect fragile quantum information and build fault-tolerant quantum computers, we use quantum error correction (\emph{QEC})~\cite[p. 46998]{Li}.
QEC is achieved by using special codes to encode quantum information~\cite[p. 113]{deBrito}.
Most codes are based on redundancy, which means the introduction of multiple qubits, instead of just one, that carry the information~\cite[p. 113]{deBrito}.

\subsection{Background}
\label{subsec:background}

This examination paper was written in the context of the lecture \emph{Quantum Software Development}, which was held by \emph{Prof. Dr. Sabine Tornow} in the summer semester of 2020 at the \emph{Munich University of Applied Sciences}.

\subsection{Description}

\citeA[p. 46998]{Li} mentions that quantum error correction is already achievable on platforms such as \emph{IBM Q Experience}~\cite{IBMQExperience}.
Using that platform this paper describes an attempt to correct bit-flip errors in a quantum circuit performing a quantum Fourier transform on a real quantum device.
Before testing on the actual computer provided by the Cloud service, the needed circuit has to be modelled and tested using a Simulator.
IBM provides a library \texttt{Qiskit} which is usable with the programming language \texttt{Python} and offers the needed Simulator as well as an interface to use the IBM Q Experience platform~\cite{Qiskit}.

\subsection{Structure}
\label{subsec:structure}

To give an overview over the structure of the paper, the following sections are quickly described.

The paper starts with a quick recapitulation about the \textbf{basic concepts}, namely \emph{quantum error correction}, \emph{quantum Fourier transform} and \textbf{related work}.
Afterwards the \textbf{used methods and tools} are introduced with which the quantum circuit is modelled and tested, followed by the actual \textbf{implementation}.
The resulting circuit is then tested and results from the simulator as well as the real quantum computer \textbf{analyzed}.

A \textbf{conclusion} about the resulting paper is drawn at the end with some \emph{proposals for further studies}.
