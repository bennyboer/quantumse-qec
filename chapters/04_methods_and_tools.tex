\section{Methods and tools}
\label{sec:methods-and-tools}

As a reminder, we will try to correct bit-flip errors of a quantum Fourier transform circuit.
Therefore, some tools and methods are selected and presented in this section.

\subsection{Qiskit}
\label{subsec:qiskit}

Since we are already familiar with the quantum computing software development framework \texttt{Qiskit}, it is an obvious choice.
Its tight coupling with IBM Quantum Experience and thus accessible quantum devices allow testing quantum circuits in a real environment~\cite{IBMQAccount}.
Besides that the framework offers a simulator capable of testing the created code as well as adding a noise model to imitate real quantum computer behavior, which we want to fight using our quantum error correction solution~\cite{QiskitNoiseModel}.

The specific \texttt{Qiskit} version used in this paper is listed in the appendix~\ref{subsec:qiskit-version}.

\subsection{Python}
\label{subsec:python}

Qiskit is a library for the programming language Python.
Thus we have to use it.
Additionally, Pythons popularity soared in the last years especially in scientific areas, making it one of the most popular programming languages~\cite{StackOverFlowDevSurveyTechnolgies}.
Thus, most people reading this paper should easily be capable of understanding the included code snippets, improving accessibility.

The precise version of the language is \texttt{Python 3.8.2} which is used throughout this paper.
There is no other reason for choosing that exact version other than \texttt{Qiskit} is supporting it and it is an up-to-date version of the language at the time of writing.

\subsection{JupyterLab}
\label{subsec:jupyter-lab}

\texttt{JupyterLab} is a web-based user interface for Project Jupyter~\cite{JupyterLabDocs}.
It is an open-source project to support interactive data science and scientific computing~\cite{ProjectJupyter}
One of its components called \emph{Jupyter Notebooks} will be used to quickly write and test Python code with additional support for LaTeX and Markdown to document that code more clearly~\cite{JupyterLabOverview}.

In addition, the code is also included in the annex to this document in order to ease the reproduction of the work of this paper.

\subsection{QFT circuit}
\label{subsec:qft-circuit}

So far we have not discussed what specific circuit is planned to be implemented.
The QFT is transforming from the computational basis in the Fourier basis where we store numbers using various rotations around the Z-axis of the individual qubits~\cite{QiskitTBQFT}.
When measuring the result we would just receive \(0\) or \(1\) with probability \(\frac{1}{2}\) (neglecting noise errors).
There is the quantum phase estimation algorithm that would estimate that phase, but for the sake of just trying to run error correction we will just transform in the computational basis again using the \emph{inverse} QFT circuit.
Afterwards we should be able to measure the input again.
Thus, we can easily validate whether errors happened.
