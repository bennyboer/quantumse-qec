\section{Related work}
\label{sec:related-work}

In the last section some basic concepts about quantum error correction were introduced.
For this paper we do not need more.
But there is certainly more about that comparatively new topic in the quantum domain.
This section will provide an overview of past and current developments in QEC.

It all began the last 90s where the noise has been discretized into discrete error models, namely bit- and phase-flip errors~\cite[p. 46999]{Li}.
Shor has developed a quantum error correction code (QECC) that successfully applies the previously reviewed bit-flip and phase-flip error circuits to protect against any type of error with 9 qubits~\cite[p. 46999]{Li}.
That event was followed by Calderbank, Shor and Steane each constructing the so-called CSS codes which were adopted from the theory of classical linear codes~\cite[Chapter: Quantum Error Correction, p. 24]{Preskill}.
It was discovered that the minimum amount of qubits needed to protect a single qubits information against any error is five~\cite[p. 46999]{Li}.

Since there seems to be no improvement to protect against any error, scientists moved on to protect against specific errors instead, thereby reducing the amount of qubits even further~\cite[p. 46999]{Li}.
That is especially useful when the error-source is known and thus can be excluded.

Discrete error models are not like the real world and rely on idealized scenarios.
For example they simplify that errors occur on qubits independently, thus there is currently increasing effort in implementing continuous error models.~\cite[p. 46999]{Li}.
