\section{Summary}
\label{sec:summary}

After having observed some interesting results we want to quickly sum up the contents of the paper followed by some proposals for further studies.

\subsection{Conclusion}
\label{subsec:conclusion}

We have been implementing a quantum circuit trying to error correct bit-flips over several repeated QFT circuits using the Framework Qiskit.
While the Qiskit simulator showed good results when manually as well as automatically configured from a given backend, current IBMQ backends seem to be still too noisy for the proposed circuit.
Additionally we are able to say that the generated noise model from Qiskit is not estimating properly - at least for the proposed circuit.

\citeA[p. 1]{tannu2018case} state that we are currently dealing with \emph{Noisy Intermediate Scale Quantum computers} that do not have the capacity to utilize QEC due to the limited amount of qubits available.
That said we could experience a little improvement with the "Paris" backend although we were only correcting bit-flip errors.
They are correct as having proper QEC codes requires a tremendous bigger amount of qubits than we have available today.

Additionally, \citeA[p. 3]{tannu2018case} write that the overall error rate is usually dominated by the gate errors from which we have a lot in the proposed circuit.
Especially the \texttt{CNOT} gate seems to have high error rates which we utilize extensively in the circuit to "read" the current qubit state to the ancillary qubits~\cite[p. 3]{tannu2018case}.
That might be another factor that leads to the observed strong noisy results.

\subsection{Proposals for further studies}
\label{subsec:proposals-for-further-studies}

In summary we find several proposals that might be worth studying in the future:

\begin{itemize}
    \item Since we rely on \texttt{CNOT} gates it would be interesting to remove them from the circuit and afterwards observe what happens to the "Only merged results" histogram.
    \item Qiskit allows us to directly select and assign specific qubits of a real quantum backend that we want to use.
    That way we might be able to use only qubits that have low error rates.
\end{itemize}
