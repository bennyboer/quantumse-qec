\section{QFT circuit with Qiskit}
\label{sec:qft-circuit-qiskit}

\subsection{Necessary imports}
\label{subsec:qft-circuit-qiskit-necessary-imports}

\begin{python}
# Qiskit
from qiskit import __qiskit_version__
from qiskit import ClassicalRegister, QuantumRegister, QuantumCircuit
from qiskit import execute, Aer
from qiskit import IBMQ
from qiskit.visualization import plot_histogram
from qiskit.providers.ibmq import least_busy

# Plotting library
import matplotlib.pyplot as plt

# Math library
import numpy as np
\end{python}

\subsection{Qiskit version}
\label{subsec:qiskit-version}

The used Qiskit version can be received by calling the following.

\begin{python}
print(__qiskit_version__)

>>> {'qiskit-terra': '0.14.1',
'qiskit-aer': '0.5.2',
'qiskit-ignis': '0.3.0',
'qiskit-ibmq-provider': '0.7.2',
'qiskit-aqua': '0.7.1',
'qiskit': '0.19.3'}
\end{python}

\subsection{QFT function}
\label{subsec:qft-function}

\begin{python}
def qft(n):
    """
    Created a quantum Fourier transform on the passed quantum circuit.
    Pass [n] as the number of qubits the circuit should be able to deal with.
    """

    qc = QuantumCircuit(n)

    # Create core QFT circuit
    for i in range(n - 1, -1, -1):
        # Apply Hadamard gate on the i'th qubit (most significant qubit)
        qc.h(i)

    # Apply controlled rotation around the Z-axis (CU1-Gate)
    for j in range(i):
        angle = np.pi / 2 ** (i - j)

        qc.cu1(theta=angle, control_qubit=j, target_qubit=i)

        qc.barrier() # Barrier for better readability of the resulting circuit graph

    # Add ending swaps
    for i in range(n // 2):
        qc.swap(i, n - 1 - i)

    return qc
\end{python}

\subsection{Inverse QFT function}
\label{subsec:inverse-qft-function}

\begin{python}
def inverse_qft(n):
    """
    Create a inverse quantum Fourier transform on the passed quantum circuit.
    Pass [n] as the number of qubits the circuit should be able to deal with.
    """

    return qft(n).inverse()
\end{python}

\subsection{Encoding and decoding number in a circuit}
\label{subsec:encoding-decoding-number-in-circuit}

\begin{python}
def encode_num(num):
    """
    Encode the passed decimal integer number in an quantum circuit.
    """

    remainders = []
    while num > 0:
        remainder = num % 2
        num = num // 2

        remainders.append(remainder)

    qc = QuantumCircuit(len(remainders))

    for i in range(len(remainders)):
        if remainders[i] == 1:
            qc.x(i)

    return qc


def decode_num(n):
    qc = QuantumCircuit(n, n)

    qc.barrier()

    qc.measure(range(n), range(n))

    return qc
\end{python}
