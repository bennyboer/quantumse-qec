\section{Error corrected circuit source code}
\label{sec:error-corrected-circuit-appendix}

\subsection{Modified QFT circuit functions}
\label{subsec:modified-qft-circuit-functions}

\begin{python}
def to_bit_representation(num):
    """
    Convert the passed decimal integer number to a bit representation.
    For example 6 will be transformed to a list: [1, 1, 0].

    Parameters:
    [num]: A decimal integer number to encode to a bit representation

    Returns:
    The passed decimal integer number as a bit list
    """

    remainders = []
    while num > 0:
        remainder = num % 2
        num = num // 2

        remainders.append(remainder)

    return remainders


def encode_bits(bits, qc, offset=0):
    """
    Encode the passed decimal integer number given as bits list in the passed quantum circuit.

    Parameters:
    [bits]: Bit list (e. g. [1, 1, 0] for the decimal number 6) to encode
    [qc]: Quantum Circuit to encode the bits in
    [offset]: The offset to apply in the passed circuit
    """

    bit_count = len(bits)

    # Encode number in circuit
    for i in range(bit_count):
    if bits[i] == 1:
    qc.x(offset + i)


    def qft(size, circuit, offset=0):
    """
    Add quantum Fourier transform gates for the passed [circuit] at the passed [offset]
    with the given [size].

    Parameters:
    [size]: How many qubits the QFT should be able to deal with
    [circuit]: Quantum circuit to apply QFT gates to
    [offset]: The offset to apply the gates in the passed circuit to
    """

    # Create core QFT circuit
    for i in range(size - 1, -1, -1):
        # Apply Hadamard gate on the i'th qubit (most significant qubit)
        circuit.h(offset + i)

    # Apply controlled rotation around the Z-axis (CU1-Gate)
    for j in range(i):
        angle = np.pi / 2 ** (i - j)

        circuit.cu1(theta=angle, control_qubit=offset + j, target_qubit=offset + i)

    # Add ending swaps
    for i in range(size // 2):
        circuit.swap(offset + i, offset + size - 1 - i)


def inverse_qft(size, circuit, offset=0):
    """
    Add the inverse quantum Fourier transform gates for the passed [circuit]
    at the given [offset] with the specified [size].

    Parameters:
    [size]: How many qubits the inverse QFT should be able to deal with
    [circuit]: Quantum circuit to apply inverse QFT gates to
    [offset]: The offset to apply the gates in the passed circuit to
    """

    # Add starting swaps
    for i in range(size // 2):
        circuit.swap(offset + i, offset + size - 1 - i)

    # Create core QFT circuit
    for i in range(size):
        # Apply controlled rotation around the Z-axis (CU1-Gate)
        for j in range(i - 1, -1, -1):
            angle = - np.pi / 2 ** (i - j)

            circuit.cu1(theta=angle, control_qubit=offset + j, target_qubit=offset + i)

    # Apply Hadamard gate on the i'th qubit (most significant qubit)
    circuit.h(offset + i)
\end{python}

\subsection{Circuit building code}
\label{subsec:circuit-building-code}

\begin{python}
def prepare_circuit(qubits_per_circuit, repetitions=1):
    """
    Prepare the quantum circuit needed for QFT with repetition code QEC.

    Parameters:
    [qubits_per_circuit]: Number of qubits per repetition
    [repetitions]: Number of repetitions to make for QEC

    Returns:
    - The prepared quantum circuit
    - A list of quantum registers
    - A list of ancilla quantum registers
    - A list of classical registers
    - A list of ancilla classical registers
    """

    # Add QFT registers for each repetition
    qrs = [] # Quantum registers
    crs = [] # Classical registers

    for r in range(repetitions):
        label = chr(ord('a') + r)

        qr = QuantumRegister(qubits_per_circuit, label)
        cr = ClassicalRegister(qubits_per_circuit, f'{label}_{{classical}}')

        qrs.append(qr)
        crs.append(cr)


    # Add ancilla registers for each repetition
    qars = [] # Quantum ancilla registers
    cars = [] # Classical ancilla registers

    ancilla_qubit_count_per_qubit = (repetitions - 1)

    for q in range(qubits_per_circuit):
        qr = QuantumRegister(ancilla_qubit_count_per_qubit, f'q{q}_{{ancilla}}')
        cr = ClassicalRegister(ancilla_qubit_count_per_qubit, f'q{q}_{{classical}}')

        qars.append(qr)
        cars.append(cr)

    return QuantumCircuit(*qrs, *crs, *qars, *cars), qrs, crs, qars, cars


def prepare_rc_qec_qft_circuit(num_to_encode, repetitions=1, with_barriers=True):
    """
    Prepare the full QEC QFT circuit.

    Parameters:
    [num_to_encode]: Decimal integer number to encode in the circuit
    [repetitions]: Number of repetitions for the repetition QEC
    """

    # Step 1: Encode the decimal integer number to a bit list
    enc_num_bits = to_bit_representation(NUM_TO_ENCODE)

    # Step 2: Prepare the quantum circuit including ancilla qubits
    qc, qrs, crs, qars, cars = prepare_circuit(len(enc_num_bits), repetitions=repetitions)

    # Step 3: Encode the bits in the quantum circuit
    for r in range(repetitions):
        encode_bits(enc_num_bits, qc, offset=r * len(enc_num_bits))

    if with_barriers:
        qc.barrier()

    # Step 4: Add the QFT gates to the circuit
    for r in range(repetitions):
        qft(len(enc_num_bits), qc, offset=r * len(enc_num_bits))

    if with_barriers:
        qc.barrier()

    # Step 5: Add the inverse QFT gates to the circuit
    for r in range(repetitions):
        inverse_qft(len(enc_num_bits), qc, offset=r * len(enc_num_bits))

    if with_barriers:
        qc.barrier()

    # Step 6: Add CNOTs to apply the QFT result qubits on our ancilla qubits
    for a in range(len(qars)): # For each ancilla qubit register
        qar = qars[a]

        for offset in range(2):
            for i in range(qar.size):
                ancilla_qubit = qar[i]
                qubit = qrs[i + offset][a]

                qc.cx(control_qubit=qubit, target_qubit=ancilla_qubit)

        if with_barriers:
            qc.barrier()

    # Step 7: Add measurements for the ancilla qubits
    for i in range(len(qars)):
        qar = qars[i]
        car = cars[i]

        qc.measure(qar, car)

    if with_barriers:
        qc.barrier()

    # Step 8: Add measurements for the qft results
    for i in range(len(qrs)):
        qr = qrs[i]
        cr = crs[i]

        qc.measure(qr, cr)

    return qc
\end{python}

\subsection{Decoding result from circuit}
\label{subsec:appendix-decoding-result-circuit}

\begin{python}
def decode_result(result_str, error_correction=True, exclude_error_results_during_merge=False):
    """
    ### Description ###
    Decode the passed result string ([result_str]) from the quantum computation in the following form:
    'ZZ YY XX CCC BBB AAA' where either letter of A, B, C, X, Y and Z can be either 0 or 1.

    ### String parts meanings ###
    AAA: Measurement result of the a-register
    BBB: Measurement result of the b-register
    CCC: Measurement result of the c-register
    XX: Measurement result of the ancilla bits for the first qubit of the a-, b- and c-registers
    YY: Measurement result of the ancilla bits for the second qubit of the a-, b- and c-registers
    ZZ: Measurement result of the ancilla bits for the third qubit of the a-, b-, and c-registers

    ### Ancilla bits meaning (ZZ, YY and XX) ###
    00: No bit-flip happened
    01: Bit-flip in register a
    10: Bit-flip in register c
    11: Bit-flip in register b

    ### Arguments ###
    - result_str: The result string to decode
    - error_correction: Whether to perform error correction
    - exclude_error_results_during_merge: Whether to exclude results where errors occurred on during merging

    ### Returns ###
    The decoded bit representation of the initial number fed into the quantum circuit
    with error correction using the ancilla bits and repetition already applied.

    ### Example ###
    Input: '10 01 00 010 110 010'

    ZZ = 10 -> means that the third qubit of register c has suffered from a bit-flip
    XX = 01 -> means that the second qubit of register a has suffered from a bit-flip
    YY = 00 -> means not bit flip happened in the first qubit

    Corrected CCC = 110 (previously 010)
    Corrected BBB = 110 (previously 110)
    Corrected AAA = 000 (previously 010)

    Result = 110 (Merged CCC, BBB and AAA)
    """

    # Step 1: Parse result string
    parts = result_str.split()

    if len(parts) != 6:
        raise Exception(f'Wrong result string format: Expected 6 strings separated by white spaces. Instead got "{result_str}"')

    ZZ = parts[0]
    XX = parts[1]
    YY = parts[2]
    CCC = parts[3]
    BBB = parts[4]
    AAA = parts[5]

    if len(ZZ) != 2 or len(XX) != 2 or len(YY) != 2 or len(CCC) != 3 or len(BBB) != 3 or len(AAA) != 3:
        raise Exception(f'Wrong result string format: Expected string to have the form "ZZ XX YY CCC BBB AAA" where each letter represents either 0 or 1')

    def to_bit_array(s):
        l = list(s)

        for i in range(len(l)):
            ch = l[i]

            if ch != '0' and ch != '1':
                raise Exception(f'Wrong result string format: Expected string to contain only 0s and 1s (separated by white spaces)')

            l[i] = int(ch)

        return l


    ZZ = to_bit_array(ZZ)
    XX = to_bit_array(XX)
    YY = to_bit_array(YY)
    CCC = to_bit_array(CCC)
    BBB = to_bit_array(BBB)
    AAA = to_bit_array(AAA)

    # Step 2: Correct bit-flips indicated by the error syndrome (ancilla bits)
    def correct_bit_flip(err_syndrome, idx, regs):
        reg_idx = None
        if err_syndrome[0] == 0:
            if err_syndrome[1] == 0: # No Bit-flip
                return None
            else: # Bit-flip in register a
                reg_idx = 2
        else:
            if err_syndrome[1] == 0: # Bit-flip in register c
                reg_idx = 0
            else: # Bit-flip in register b
                reg_idx = 1

        reg = regs[reg_idx]
        reg[idx] = 0 if reg[idx] == 1 else 1

        return idx

    regs = [CCC, BBB, AAA]
    corrected_reg_indices = []
    if error_correction:
        reg_idx = correct_bit_flip(ZZ, 0, regs)
        if reg_idx is not None:
            corrected_reg_indices.append(reg_idx)

        reg_idx = correct_bit_flip(YY, 1, regs)
        if reg_idx is not None:
            corrected_reg_indices.append(reg_idx)

        reg_idx = correct_bit_flip(XX, 2, regs)
        if reg_idx is not None:
            corrected_reg_indices.append(reg_idx)

    # Step 3: Merge results
    result = [0] * 3

    for i in range(len(result)):
        zeros = 0
        ones = 0
        for a in range(len(regs)):
            reg = regs[a]
            if not exclude_error_results_during_merge or not a in corrected_reg_indices:
                if reg[i] == 0:
                    zeros += 1
                else:
                    ones += 1

        result[i] = 1 if ones > zeros else 0

    return result
\end{python}

\subsection{Source code used to generate a histogram of the results}
\label{subsec:code-histogram-results}

\begin{python}
def plot_histograms(counts):
    fig, axes = plt.subplots(2, 2, figsize=(14,8))

    # Plot without performing any error correction
    result_map = {}
    for key, value in counts.items():
        parts = key.split()

        for i in range(3, 6, 1):
            r = parts[i]

        if r not in result_map:
            result_map[r] = value
        else:
            result_map[r] += value

    plot_histogram(result_map, ax=axes[0][0])
    axes[0][0].set_title("No error correction", fontsize=20)

    # Plot without performing bit-flip correction but allow merging
    result_map = {}
    for key, value in counts.items():
        r = "".join(map(lambda x: str(x), decode_result(key, error_correction=False)))

        if r not in result_map:
            result_map[r] = value
        else:
            result_map[r] += value

    plot_histogram(result_map, ax=axes[0][1])
    axes[0][1].set_title("Only merging results", fontsize=20)

    # Plot with error-correction
    result_map = {}
    for key, value in counts.items():
        r = "".join(map(lambda x: str(x), decode_result(key)))

        if r not in result_map:
            result_map[r] = value
        else:
            result_map[r] += value

    plot_histogram(result_map, ax=axes[1][0])
    axes[1][0].set_title("Error correction", fontsize=20)

    # Plot without allowing bit-flip results to be merged
    result_map = {}
    for key, value in counts.items():
        r = "".join(map(lambda x: str(x), decode_result(key, exclude_error_results_during_merge=True)))

        if r not in result_map:
            result_map[r] = value
        else:
            result_map[r] += value

    plot_histogram(result_map, ax=axes[1][1])
    axes[1][1].set_title("Only merge results without error", fontsize=20)

    fig.tight_layout(pad=2.5)

    fig.savefig('out/test-histogram-qec-results.pdf', transparent=True)
\end{python}
